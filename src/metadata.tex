% Description: Metadata for the plain template
%
% Warning: this file is generated automatically
% from the paper-meta.yaml file. If you really want to edit
% this file, please mirror the changes to paper-meta.yaml.

\title{N-polyregular functions arise from well-quasi-orderings}
\author{%
        Aliaume Lopez\thanks{University of Warsaw}%
    % then mail address with a letter symbol 
    ~~\href{mailto:ad.lopez@uw.edu.pl}{\Letter}%
    }


\newcommand{\makeabstract}{
\begin{abstract}
    A fundamental construction in formal language theory is the
    Myhill-Nerode congruence on words, whose finitedness characterizes
    regular language. This construction was generalized to functions
    from \(\Sigma^*\) to \(\mathbb{Z}\) by Colcombet, Douéneau-Tabot,
    and Lopez to characterize the class of so-called
    \(\mathbb{Z}\)-polyregular functions. In this paper, we relax the
    notion of equivalence relation to quasi-ordering in order to study
    the class of functions from \(\Sigma^*\) to \(\mathbb{N}\). The
    analogue of having a finite index is then being a
    well-quasi-ordering. This provides a canonical object to describe
    \(\mathbb{N}\)-polyregular functions, which was lacking prior to
    this work.
\end{abstract}
}
