%! TeX program = xelatex
%! lang = en-US

\begin{proofof}{non-canonical-transd:ex}
    Let us prove that the transducer of
    \cref{non-canonical-transd:fig:counter-free}
    computes $f$.
    We prove by induction on $w$
    that $A(q_0, w) = f(w)$,
    and that $A(q_1, w) = f(aw)$.
    To that end, let us first remark that $f(\varepsilon) = 1$,
    and $f(au) = \card{u}$.
    
    When $w = \varepsilon$, $A(q_0, \varepsilon) = F(\varepsilon) = 1 = f(\varepsilon)$.
    Similarly, $A(q_1, \varepsilon) = F(q_1) = 0 = f(a)$.

    Assume that $w = au$. Then:
    \begin{align*}
        A(q_0, w) = A(q_0, au) &= \lambda(q_0, a)(u) + A(q_1, u) \\ 
                               &= 0 + A(q_1,u) \\
                               &= f(au) & \text{by induction hypothesis} \\
                               &= f(w) 
    \end{align*}
    Similarly,
    \begin{align*}
        A(q_1, w) = A(q_1, au) &= \lambda(q_1, a)(u) + A(q_1, u) \\ 
                               &= 1 + A(q_1,u) \\
                               &= 1 + f(au) & \text{by induction hypothesis} \\
                               &= 1 + \card{u} \\
                               &= \card{au} \\
                               &= f(aau) \\
    \end{align*}

    The proof for the other automaton is similar. The
    key ingredient in the induction hypothesis is that if $\card{u} \geq 1$, then:
    $f(au) - f(u) = 1$ and otherwise $f(a) - f(\varepsilon) = -1$.
    Hence, $f(au) - f(u) = 1 - 2 \times \ind{\card{u} = 0} = \lambda(q_1, a)(u)$.
\end{proofof}
\section{Proofs of section \ref{beyond-commutative:sec}}

\begin{remark}
    \label{good-residual-ordering:fact}
    Let $k \in \Nat$, and let $f \colon \Sigma^* \to \Nat$. Then,
    $(\resleq{f}{k})$ is a quasi-ordering, satisfying the following
    extra properties:
    \begin{enumerate}
        \item For all $u,v,w \in \Sigma^*$, $u \resleq{f}{k} v$
            implies $uw \resleq{f}{k} vw$,
        \item If $u \resleq{f}{k} v$ and $v \resleq{f}{k} u$,
            then $\app{f}{u} = \app{f}{v}$.
    \end{enumerate}
\end{remark}


\begin{fact}
    \label{unique-res-transducer:fact}
    Let $f \colon \Sigma^* \to \Nat$ and $k \in \Nat$.
    Then $f$ has at most one \kl{$k$-residual transducer}.
\end{fact}
\begin{proof}
    Let $\aTransd_1$ and $\aTransd_2$ be two
    \kl{$k$-residual transducers} for $f$.
    The two initial states must be $\varepsilon$.
    Let us prove by induction on $u \in \Sigma^*$ that
    $\delta_1(\varepsilon, u) = \delta_2(\varepsilon, u)$
    and that $Q_1$ equals $Q_2$ over prefixes of $u$.
    This will prove that 
    $Q_1 = Q_2$, hence that $\aTransd_1 = \aTransd_2$.

    Let $u \in \Sigma^* \cap Q_1 \cap Q_2$ and $a \in \Sigma$, $v_1 \in Q_1$ be
    defined as $v \defined \delta_1(u,a)$, and $v_2 \defined \delta_2(u,a)$.
    Remark that by induction hypothesis, for all $v \prefle u$, $v \in Q_1 \cap
    Q_2$. If $\delta_1(u,a) = Q_1$, it means that for all $v \in Q_2$ such that
    $v \prefle ua$, we have $\neg( v \resleq{f}{k} ua )$. The only possible
    transition in $\aTransd_2$ is therefore $\delta_2(u,a) = ua$, and $ua \in
    Q_2$. Similarly, if $\delta_1(u,a) \prefleq u$, then $\delta_2(u,a) =
    \delta_1(u,a)$ by definition of $\delta_2$ as a maximum.
\end{proof}


\begin{fact}
    \label{q-o-prefix-cool:fact}
    Let $f \colon \Sigma^* \to \Nat$ and $k \in \Nat$.
    At each step of the \texttt{while loop}
    of \cref{residual:algo}, the sets
    $Q$ and $O$ are such that
    \begin{enumerate}
        \item $Q \cup O$ is a \kl{downwards closed} subset of 
            $\Sigma^*$ for $\prefleq$;
        \item elements in $O$ are pairwise incomparable
            for $\prefleq$, and are maximal
            for $\prefleq$ inside $Q \cup O$.
    \end{enumerate}
\end{fact}
\begin{proof}
    Let us write $Q_i$ and $O_i$ for the value of the variables
    $Q$ and $O$ at step $i$ of the \texttt{while loop}.
    We prove the desired property by induction on $i$.

    For $i=0$, the property is true because
    $Q_0 = \set{\varepsilon}$ and $O_0 = \setof{a}{a \in \Sigma}$.

    For $i+1$. Either the \texttt{if} branch was taken, in which case $Q_{i+1}
    \cup O_{i+1} = (Q_i \cup O_i) \setminus \set{u}$ for some $u \in O_i$. This
    set remains \kl{downwards closed}, and elements in $O_{i+1}$ remain maximal
    elements. 

    If the \texttt{else} branch was taken, then there exists $u \in O_i$ such
    that $Q_{i+1} = Q_i \cup \set{u}$ and $O_{i+1} = O_i \setminus \set{ u }
    \cup \setof{ ua }{ a \in \Sigma}$. We conclude that $Q_{i+1} \cup O_{i+1} =
    Q_i \cup O_i \cup \setof{ ua }{a \in \Sigma}$ continues to be \kl{downwards
    closed} for $\prefleq$. Let $v \in Q_{i+1} \cup O_{i+1}$ be such that $ua
    \prefleq v$ for some $a \in \Sigma$. Then $u \prefleq v$, and $u = v$ since
    $u$ was a maximal element. As a consequence, $ua$ is a maximal element for
    all $a \in \Sigma$. Assume by contradiction that $ua$ is comparable with
    some $v \in O_{i+1}$ with $ua \neq v$, it cannot be that $ua \prefleq v$ by
    the above argument, and if $v \prefleq ua$ with $v \neq ua$, then $v
    \prefleq u$ and $u = v$, which is absurd since $v \not \in O{i+1}$.
    We have concluded that $O_{i+1}$ continues to have pairwise incomparable
    elements.
\end{proof}


\begin{lemma}
    \label{correct-residual:lemma}
    If \cref{residual:algo} terminates on 
    an input $f \colon \Sigma^* \to \Nat$, $k \in \Nat$,
    then it computes the \kl{$k$-residual transducer} of $f$.
    \proofref{correct-residual:lemma}
\end{lemma}
\begin{proof}
    Because of \cref{q-o-prefix-cool:fact},
    we already know that $q_0 = \varepsilon$,
    $Q$ is a \kl{downwards closed} subset of $\Sigma^*$
    for $\prefleq$, 
    that every state of $Q$ is accessible from $q_0$.
    Notice that at every step,
    $\lambda(u,a)$ is defined as
    $\app{f}{ua} - \app{f}{\delta(u,a)}$.
    Finally, since $Q \cup O$ is a \kl{downwards closed} subset of $\Sigma^*$
    at every step,
    we have that at step $i$,
    for all $ua \in O_i$,
    $\setof{w \in Q}{w \prefleq ua} = \setof{w \in Q_i}{ w \prefleq ua}$,
    which proves that the maximum considered in the algorithm
    is indeed computing correctly.
\end{proof}

\begin{lemma}
    \label{wqo-implies-termination:lemma}
    Let $f \colon \Sigma^* \to \Nat$, and $k \in \Nat$ be such that
    every infinite, $\prefleq$-increasing sequence is \kl{good}
    in $(\Sigma^*, \resleq{f}{k})$
    (or equivalently, such that the relation $(\prefleq \Rightarrow \resleq{f}{k})$
    is \kl{well}).
    Then, \cref{residual:algo} terminates on the input $(f,k)$.
    \proofref{wqo-implies-termination:lemma}
\end{lemma}
\begin{proof}
    Assume towards a contradiction that
    \cref{residual:algo} does not terminate.
    Then, the \texttt{else} branch in the \texttt{while loop}
    must be taken infinitely often.
    This means that the set $Q$ of states grows arbitrarily large.

    Let us write $\seqof{Q_i}{i \in \Nat}$ for the set of states $Q$ at step
    $i$ of the execution of \cref{residual:algo}. Applying
    \cref{q-o-prefix-cool:fact}, we know that for all $i \in \Nat$, $Q_i$ is
    \kl{downwards closed} for $\prefleq$. Let us write $Q_\infty \defined
    \bigcup_{i \in \Nat} Q_i$. The set $Q_\infty$ is infinite, and is
    \kl{downwards closed} for $\prefleq$. As a consequence, it is an infinite
    tree with a finite branching (at most $\card{\Sigma}$), and has an infinite
    branch $\seqof{u_j}{j \in \Nat}$ thanks to König' s lemma.

    Let us prove that this infinite branch is a \kl{bad sequence} for the
    ordering $\resleq{f}{k}$.
    Let $j < p$, and assume by contradiction that $u_j \resleq{f}{k} u_p$. We
    know that $u_j \in Q_j$ and $u_p \in Q_p$. Then, at step $p-1$ of the
    algorithm, $u_j \in Q_{p-1}$, since $u_j \in Q_j \subseteq Q_{p-1}$.
    Because $u_j \prefleq u_p$ and $u_j \resleq{f}{k} u_p$,
    \cref{residual:algo} must take the \texttt{if} branch at step $p-1$. As a
    consequence, $u_p \not\in Q_{p}$, which is absurd.

    We have proven that the infinite branch is a \kl{bad sequence}
    for $\resleq{f}{k}$, which contradicts the assumption.
    Hence, \cref{residual:algo} must terminate.
\end{proof}



\begin{lemma}
    \label{n-poly-k-implies-wqo:lemma}
    Let $k \in \Nat$, and let $f \in \NPoly[k]$.
    Then, $(\Sigma^* \resleq{f}{k})$ is a \kl{well-quasi-ordering}.
\end{lemma}
\begin{proof}
    Because $f \in \NPoly[k]$, there exists
    a tuple $\vec{x}$ of first order free variables,
    $\MSO$ formulas $\seqof{\psi_i(\vec{x})}{1 \leq i \leq n}$,
    and positive coefficients $\seqof{m_i}{1 \leq i \leq n}$,
    such that
    $f = \sum_{i = 1}^n m_i \times \vcount{\varphi_i(\vec{x})}$
    \cite[Theorem 5.15]{DOUE23}.

    Let $q$ be the maximal quantifier rank of formulas $\seqof{\psi_i}{1 \leq i
    \leq n}$. To a word $u \in \Sigma^*$, we associate the vector $\MSO^q(u)$
    that maps an $\MSO$-type with $\ell \leq |\vec{x}|$ free variables to the
    number of realizations of this type in $u$.

    Let $u, v \in \Sigma^*$ such that $\MSO^q(u) \leq \MSO^q(v)$, which means
    that every $\MSO$ type (of quantifier rank $q$ and with at most $n$ free
    variables) has at least as many realizations in $v$ than it has in $u$.
    Remark that by the compositionality of $\MSO$ over words (for instance, see
    the Feferman-Vaught theorem \cite{FEVAU59,MAKOW04}), for all $\MSO^q$ types
    $t(\vec{x})$, there are finitely many $\MSO^q$ types $t_l^j(\vec{y_i}),
    t_r^j(\vec{z_i})$ with $\vec{x} = \vec{y_i} \uplus \vec{z_i}$
    for $1 \leq j \leq N_0$, such that for every
    tuple $\vec{a}$ of elements in a word $uv$, $\MSO^q(\vec{a} / uv) =
    t(\vec{x})$ if and only if there exists $1 \leq j \leq N_0$,
    such that $\vec{a} = \vec{b} \uplus \vec{c}$,
    $\MSO^q(\vec{b} / u) =
    t_l^j(\vec{y_i})$, and $\MSO^q(\vec{c} / v) = t_r^j(\vec{z_i})$.
    We write $t = t_l \odot t_r$ to signify
    that $\MSO^q(\vec{bc} / uv) = t$
    if and only if $\MSO^q(\vec{b}/u) = t_l$
    and $\MSO^q(\vec{c}/v) = t_r$.

    As a consequence, if $\MSO^q(u) \leq \MSO^q(v)$, then 
    for all $w \in \Sigma^*$:
    \begin{align*}
        & f(vw) - f(uw) \\
        &= 
        \sum_{i = 1}^n m_i
        \left[
            \vcount{\phi_i(\vec{x})} (vw) -
            \vcount{\phi_i(\vec{x})} (uw)
        \right] \\
        &= 
        \sum_{i = 1}^n
        m_i
            \sum_{\phi_i \in t(\vec{x})}
        \left[
            \vcount{t(\vec{x})}(vw)
            -
            \vcount{t(\vec{x})}(uw)
        \right] \\
        &= 
        \sum_{i = 1}^n
        m_i
        \sum_{1 \leq j \leq N_0}
        \sum_{\phi_i \in t_l^j(\vec{y}) \odot t_r^j(\vec{z})}
        \underbrace{
        \left[
            \vcount{t_r^j(\vec{y})}(v)
            -
            \vcount{t_r^j(\vec{y})}(u)
        \right] 
    }_{ \in \Nat }
            \times 
            \vcount{t_l^j(\vec{z})}(w)
    \end{align*}

    We have proven that if $\MSO^q(u) \leq \MSO^q(v)$, then $u \resleq{f}{k}
    v$. Recall that $\Nat^p$ is a \kl{well-quasi-ordering} when endowed with
    the product ordering, and therefore that $\setof{\MSO^q(u)}{u \in
    \Sigma^*}$ is a \kl{well-quasi-ordering}.

    Let $\seqof{u_i}{i \in \Nat}$ be an infinite sequence of $\Sigma^*$.
    Without loss of generality, one can assume that for all $i \neq j$, $u_i
    \equiv_k u_j$, i.e., that the difference $\app{f}{u_i} - \app{f}{u_j}$
    belongs to $\ZPoly[k-1]$, since the latter has finite index. Thanks to the
    above remarks, there exists $i < j$ such that $\MSO^q(u_i) \leq
    \MSO^q(u_j)$. As a consequence, $g \defined \app{f}{u_j} - \app{f}{u_i} \in
    \NPoly$, and therefore $g \in \NPoly[k-1]$. We have proven that there
    exists $i < j$ such that $u_i \resleq{f}{k} u_j$.
\end{proof}



\begin{proofof}{non-commutative-npoly:thm}
    \cref{n-poly-1-transd:item} implies \cref{n-poly-k:item} by
    definition. Then,
    \cref{n-poly-k:item} implies \cref{n-poly-wqo:item} by
    \cref{n-poly-k-implies-wqo:lemma}.
    The implication \cref{n-poly-wqo:item} $\Rightarrow$ \cref{n-poly-well:item}
    is obvious.
    Then, \cref{wqo-implies-termination:lemma} proves
    that \cref{n-poly-well:item} implies \cref{n-poly-residual:item}.
    Finally, because a \kl{$k$-residual transducer} is a \kl{$\NPoly[k-1]$-transducer},
    \cref{n-poly-residual:item} implies \cref{n-poly-1-transd:item}.
\end{proofof}

\begin{proofof}{aperiodic-iff-residual:lem}
    It is clear that $\NSF \subseteq \ZSF$, and known that if $f \in \ZSF$
    then it is \kl{ultimately polynomial}. Furthermore, if the \kl{$0$-residual
    transducer} of $f$ is \kl{counter-free}, then $f \in \NSF$
    by definition of $\NSF$.

    Assume that $f$ is \kl{ultimately polynomial}, let us prove that the
    \kl{$0$-residual transducer} of $f$ is \kl{counter-free}. 
    Note that because $f \in \NPoly[0]$,
    $u \resleq{f}{0} v$ if and only if $\app{f}{u} = \app{f}{v}$.
    In particular, in a \kl{$0$-residual transducer} of $f$,  two states that
    represent the same residual must be incomparable for the prefix relation.

    Let $(q,w^n)$ be
    a counter with $n \geq 1$. This means that $\delta(q, w^n) = q$ in the
    automaton, and implies that $q \resleq{f}{0} qw^n$, hence that $\app{f}{q}
    = \app{f}{qw^n}$. Because $f$ is \kl{ultimately polynomial},
    we know that $\app{f}{qw^n} = \app{f}{qw^{n+1}}$,
    hence that $\app{f}{qw} = \app{f}{q}$.

    Let us write $t \defined \delta(q,w) = \delta(q,w^{n+1})$. We know that
    $\app{f}{q} = \app{f}{t}$. Assume by contradiction that $t$ and $q$ are
    incomparable for the prefix relation. Let us split $w = w_1 w_2$ where
    $w_1$ is the shortest prefix of $w$ such that $s_0 \defined \delta(q,w_1)$
    is an ancestor of $q$ and of $t$ for the prefix relation, it must exist
    because $\delta(q,w_1 w_2) = t$.

    Now, consider $s_1 \defined \delta(t, w_1)$. Assume by contradiction that
    $s_0$ is not comparable with $s_1$ for the prefix relation. Then, consider
    the smallest prefix $v$ of $w_1$ such that $\delta(t, v)$ is a strict
    prefix of $s_0$. It must exist, otherwise $s_0$ is always a prefix of
    $s_1$. Because $\app{f}{t} = \app{f}{q}$, we conclude that $\app{f}{tv} =
    \app{f}{qv}$. However, this contradicts the minimality of $w_1$, since
    $\delta(t,v)$ is an ancestor of $q$ and $t$.

    We have proven that $s_0$ and $s_1$ are comparable, hence they are equal,
    since $\app{f}{s_1} = \app{f}{t w_1} = \app{f}{q w_1} = \app{f}{s_0}$.
    Finally, we have proven that $\delta(q, w_1) = s_0$, $\delta(s_0, w_2) =
    t$, and $\delta(s_0, w_2) = \delta(t, w_1w_2) = q$ which is absurd.
    As a consequence $t$ and $q$ were comparable for the prefix relation,
    hence equal, and therefore $\delta(q, w) = q$.
\end{proofof}





\begin{proofof}{sf-no-periods-on-sequences:lemma}
    Let $f \in \NSF[k]$, $q \in \Nat$, and 
    write $f = \sum_{i=1}^n m_i \times \vcount{\phi_i(\vec{x})}$, where
    $\phi_i \in \FO$ has quantifier rank at most $q$
    and $\card{\vec{x}} = k$
    \cite[Theorem 7.10]{DOUE23}.

    As in the proof of \cref{n-poly-k-implies-wqo:lemma}, we are going to
    assign a tuple of integers to a word $u \in \Sigma^*$ by counting the
    number of realizations of each $\FO^q$ type of at most $k$ variables in
    $u$. To that end, let us write $\FO^q(u)$ this vector.

    First, let us notice that it suffices to prove that for some $n \in \Nat$,
    $\FO^q(uw^n) \leq \FO^q(uw^{n+1})$ to conclude, since $\app{f}{uw^{n+1}} -
    \app{f}{uw^n}$ will be obtained as a positive combination of counting
    first-order types in the argument.

    Let $t$ be a first order type with at most $k$ free variables and
    quantifier rank at most $q$. The map $g_t \colon X \mapsto
    \vcount{t}(uw^X)$ is a \kl{commutative} \kl{star-free $\Nat$-polyregular
    function}. As a consequence, there exists $N_0$ and a polynomial $P \in
    \CorrectPoly$ such that for all $X \geq N_0$, $g_t(X) = P(X)$. Now, because
    $P \in \CorrectPoly$, there exists a $K \in \Nat$ such that
    $\translate{K}(\Diff{1}{P}) \in \Nat[X]$
    (\cref{derivation-translation:lem}), and in particular $g_t(X + N_0 + K+1)
    - g_t(X + N_0 + K) \in \Nat$, for all $X \geq 0$.

    Because there are finitely many non-equivalent $\FO^q$ types with at most 
    $k$ free variables, we can take the maximum of the $K$'s obtained for each 
    of those, and conclude.
\end{proofof}
